\documentclass[9pt,twocolumn,twoside,lineno]{pnas-new}
% Use the lineno option to display guide line numbers if required.

% Automatic formatting of SI units
\usepackage[binary-units]{siunitx}

% Visible TODO notes
\newcommand{\todo}[1]{\textbf{\textsc{\textcolor{red}{(TODO: #1)}}}}

\templatetype{pnasresearcharticle} % Choose template 
% {pnasresearcharticle} = Template for a two-column research article
% {pnasmathematics} %= Template for a one-column mathematics article
% {pnasinvited} %= Template for a PNAS invited submission

\title{Large-scale brain simulations on the desktop using procedural connectivity}

% Use letters for affiliations, numbers to show equal authorship (if applicable) and to indicate the corresponding author
\author[a,1]{James C Knight}
\author[a]{Thomas Nowotny} 

\affil[a]{Centre for Computational Neuroscience and Robotics, School of Engineering and Informatics, University of Sussex, Brighton, United Kingdom}

% Please give the surname of the lead author for the running footer
\leadauthor{Knight} 

% Please add here a significance statement to explain the relevance of your work
\significancestatement{Authors must submit a 120-word maximum statement about the significance of their research paper written at a level understandable to an undergraduate educated scientist outside their field of speciality. The primary goal of the Significance Statement is to explain the relevance of the work in broad context to a broad readership. The Significance Statement appears in the paper itself and is required for all research papers.}

% Please include corresponding author, author contribution and author declaration information
\authorcontributions{J.K. and T.N. wrote the paper.
T.N. is the original developer of GeNN.
J.K. is currently the primary GeNN developer and was responsible for extending the code generation approach to the procedural simulation of synaptic connectivity.
J.K. performed the experiments and the analysis of the results that are presented in this work.}

\authordeclaration{The authors declare no conflict of interest.}
\correspondingauthor{\textsuperscript{1}To whom correspondence should be addressed. E-mail: J.C.Knight\@sussex.ac.uk}

% Keywords are not mandatory, but authors are strongly encouraged to provide them. If provided, please include two to five keywords, separated by the pipe symbol, e.g:
\keywords{spiking neural networks $|$ GPU $|$ high-performance computing $|$ brain simulation} 

\begin{abstract}
Large-scale simulations of spiking neural networks have become important tools in helping us improve the dynamics and, ultimately, the function of the brain.
However, even small mammals such as mice have around \num{1E12} synaptic connections~\citep{Herculano-Houzel2010}, the strengths of which are typically modelled as individual floating point values.
Expressing these values using double precision floating point would require over \SI{7}{\tera\byte} of memory and, even if half precision floating point were used, more than \SI{1}{\tera\byte} would still be required.
As such memory requirements are beyond what is plausible for a single machine, simulations of large-scale spiking neural network currently need to be distributed across large numbers of supercomputer nodes.
Large parts of such models are typically described by simple algorithms to describe connectivity and probability distributions to describe the strength of synaptic connections.
In this work, we describe our extensions to GeNN~\citep{Yavuz2016} -- our GPU-based spiking neural network simulator -- to enable it to `procedurally' generate connectivity and synaptic weights as spikes are received rather than retrieving them in memory.
We find that high-end GPUs are well-suited to this approach as they provide a large amount of raw computational power which is often under-utilised when simulating spiking neural networks due to the limited memory bandwidth available to each parallel computing element.
To demonstrate the value of this approach, we present the results of simulations of a recent model of the Macaque visual cortex consisting of \num{4.13E6} neurons and \num{24.2E9} synapses on a single GPU and show that the results are correct and the simulation runs faster than previous simulations run on using over 1000 supercomputer nodes.

\end{abstract}

\dates{This manuscript was compiled on \today}
\doi{\url{www.pnas.org/cgi/doi/10.1073/pnas.XXXXXXXXXX}}

\begin{document}

\maketitle
\thispagestyle{firststyle}
\ifthenelse{\boolean{shortarticle}}{\ifthenelse{\boolean{singlecolumn}}{\abscontentformatted}{\abscontent}}{}

% If your first paragraph (i.e. with the \dropcap) contains a list environment (quote, quotation, theorem, definition, enumerate, itemize...), the line after the list may have some extra indentation. If this is the case, add \parshape=0 to the end of the list environment.
\dropcap{W}hile there are numerous computational models of neurons~\todo{cite}, the majority of them can be simulated by solving an Ordinary Differential Equation~(ODE) every simulation timestep.
Solving ODEs is a key part of many scientific workloads and, as such, there are existing solutions to 
Additionally, if we consider the brain of a mouse, it `only' has \num{71E6} neurons which, even if we were to model them using the 4 state variable Hodgkin Huxley model~\todo{cite} with each state variable represented using a \SI{32}{\bit} type could be stored in around \SI{1}{\giga\byte}.
However, anatomical data from across mammalian species suggests that each neuron in the cortex has on average \num{8000} incoming synaptic connections~\citep{beaulieu1989number,pakkenberg2003aging,braitenberg2013cortex} and, in total, mice have approximately \num{1E12} synapses.
While simulating synaptic plasticity -- the family of mechanisms believed to be responsible for learning -- represents a further challenge, in a large-scale model, it is unlikely that learning would be enabled on \emph{all} synapses so efficiently simulating the remaining static synapses is a key challenge for large-scale brain simulation.\todo{has this been at all quantified in mice?}
Simulating such synapses essentially consists of reading a sparse `row' of synapses associated with a presynaptic spike and adding the synaptic weights of each synapse to a bin -- a form of weighted histogram update problem.
Because typical EPSP shaping functions are linear, they can then be subsequently applied to the `histogram' resulting from this process.~\todo{presumably someone first had this intuition so cite}.

Large-scale brain models are currently typically simulated on large distributed systems using software such as NEST~\citep{Gewaltig2007} or NEURON~\citep{carnevale2006neuron}.
By careful design and by distributing simulations across many thousands of nodes, such simulators can keep memory requirements of each node can kept constant~\citep{Jordan2018}.
However, 

Neuromorphic systems~\citep{Frenkel2018,Frenkel2019,Furber2014,Merolla2014,Qiao2015,Schemmel2017} have been specifically developed for simulating spiking neural networks and take inspiration from the brain to address the issues of memory bandwidth and capacity.
One particular relevant feature of the brain is that its memory elements -- the synapses -- are located throughout the system rather than being centrally located.
In neuromorphic systems, this often translates to a large proportion of each chip being dedicated to memory, either in the form of analogue `floating gates' or digital SRAM cells.
However, while SRAM is fast, it can only be fabricated at relatively low density meaning that many of these systems economize -- either by reducing the maximum number of synapses per neuron to as few as \num{256} or by reducing the precision of the synaptic weights to \num{6}~\citep{Schemmel2017}, \num{4}~\citep{Frenkel2018} or even \SI{1}{\bit}~\citep{Merolla2014,Frenkel2019}.
However, in the context of large-scale brain simulation, reducing the degree of connectivity to fit within the constraints of such a system inevitably changes its dynamics~\citep{VanAlbada2015}.
Unlike the majority of other other neuromorphic systems, SpiNNaker~\citep{Furber2014} is entirely programmable and, as well as having a relatively large amount of SRAM on each chip, also has external SDRAM distributed across the system which is typically used for the storage of synaptic connectivity.
The ratio of SDRAM bandwidth to computation was specifically designed 
It was recently demonstrated that \citep{Rhodes2019}
However, as an academic project, SpiNNaker has been 20 years in development so is using a relatively low-tech \SI{130}{\nano\metre} fabrication technology.
This means that a physically large system is required for even moderately-sized simulations.
A next generation SpiNNaker system is currently under development~\citep{Mayr2019} which aims to scale-down a similar architecture to a more modern \SI{22}{\nano\metre} fabrication technology.
Current performance estimates are that a single chip of the new system will offer equivalent performance to a entire board of current SpiNNaker system.\todo{am attempting to obtain a memory bandwidth figure which will hopefully make this argument stronger}.
Nonetheless, large-scale brain simulations will still require a large multi-chip system.
SpiNNaker 1 - \SI{200}{\mega\hertz}, \SI{600}{\mega\byte\per\second}, 16 app cores - 5 cycles per byte BUT output in SRAM.

Typical GPU architectures have relatively small amounts on on-chip memory (typically a small amount on each core configurable as either a scratch-pad or a first level-cache and a larger amount shared between all cores used as a second level cache) with the majority of the GPU die dedicated to arithmetic logic units~(ALUs).
GPUs use dedictated hardware to rapidly switch between tasks meaning that, as long as there is sufficient computation to be performed, the latency of accessing external memory can be `hidden' behind computation.
Using our previous naive metric, a `Nvidia Titan RTX'~\todo{cite} GPU has \num{4608} `CUDA cores' which can run at a sustained clock speed of \SI{1.350}{\giga\hertz} and can perform the majority of \SI{32}{\bit} operations in 1 clock cycle.
Data is delivered to these cores from high-speed GDDR6 memory with a bandwidth of \SI{672}{\giga\byte\per\second}meaning that each CUDA core needs to perform approximately 10 arithmetic operations for every byte of data accessed from main memory to successfully hide the memory latency.
Based on these characteristics, it would therefore seem that GPUs are singularly ill-suited to spiking neural network simulation.
However, in our previous work~\citep{Knight2018} we showed that, although the memory bandwidth per core is limited, the total memory bandwidth is still siginficantly higher than even the most expensive CPU meaning that moderately sized models of around \num{10E3} neurons and \num{1E9} synapses can be simulated on a single GPU with competitive speed and energy requirements.\todo{real-time not likely}
Nonetheless, individual GPUs do not have enough memory to simulate truly large-scale brain models and, although small numbers of GPUs can be connected together using the high-speed NVLink~\todo{cite} interconnect, beyond this scaling will be bound by the same communication overheads as CPU-based distributed systems.

In this work we present an approach which converts large-scale brain simulation from a problem which is memory-bound on a GPU to one where the large amounts of computational power available on GPU architectures can be used to reduce the memory and memory bandwidth requirements and make large-scale brain simulations plausible on a single workstation.

\section*{Results}
\subsection*{Procedural connectivity}
\begin{figure*}
     \centering
    \includegraphics{figures/performance_scaling}
    \caption{Performance scaling .}
    \label{fig:performance_scaling}
\end{figure*}

\subsection*{Kernel merging}
aaaaa

\subsection*{The multi-area model}
Due to lack of both computing power and sufficiently detailed connectivity data, previous models of the cortex have either focussed on modelling individual local microcircuits~\citep{Izhikevich2008,Potjans2012} at the level of individual cells or modelling multiple connected areas at a higher level of abstraction where entire ensembles of neurons are described by a small number of differential equations~\todo{find citation}.
However, data from several species~\todo{find citation} has shown that cortical activity has distinct features at both the global and local levels which can be captured by modelling interconnected microcircuits at the level of individual cells.

By using a supercomputer to simulate a model based on the latest connectivity data and The multi-scale model of the macaque visual cortex~\citep{Schmidt2018} developed by 

\section*{Discussion}
\begin{itemize}
    \item Further scaling - memory only required for neuron parameters
    \item Learning
    \item Hardware for procedural connectivity?
\end{itemize}
% 
% \section*{Guide to using this template on Overleaf}
% 
% Please note that whilst this template provides a preview of the typeset manuscript for submission, to help in this preparation, it will not necessarily be the final publication layout. For more detailed information please see the \href{http://www.pnas.org/site/authors/format.xhtml}{PNAS Information for Authors}.
% 
% If you have a question while using this template on Overleaf, please use the help menu (``?'') on the top bar to search for \href{https://www.overleaf.com/help}{help and tutorials}. You can also \href{https://www.overleaf.com/contact}{contact the Overleaf support team} at any time with specific questions about your manuscript or feedback on the template.
% 
% \subsection*{Author Affiliations}
% 
% Include department, institution, and complete address, with the ZIP/postal code, for each author. Use lower case letters to match authors with institutions, as shown in the example. Authors with an ORCID ID may supply this information at submission.
% 
% \subsection*{Submitting Manuscripts}
% 
% All authors must submit their articles at \href{http://www.pnascentral.org/cgi-bin/main.plex}{PNAScentral}. If you are using Overleaf to write your article, you can use the ``Submit to PNAS'' option in the top bar of the editor window. 
% 
% \subsection*{Format}
% 
% Many authors find it useful to organize their manuscripts with the following order of sections;  Title, Author Affiliation, Keywords, Abstract, Significance Statement, Results, Discussion, Materials and methods, Acknowledgments, and References. Other orders and headings are permitted.
% 
% \subsection*{Manuscript Length}
% 
% PNAS generally uses a two-column format averaging 67 characters, including spaces, per line. The maximum length of a Direct Submission research article is six pages and a Direct Submission Plus research article is ten pages including all text, spaces, and the number of characters displaced by figures, tables, and equations.  When submitting tables, figures, and/or equations in addition to text, keep the text for your manuscript under 39,000 characters (including spaces) for Direct Submissions and 72,000 characters (including spaces) for Direct Submission Plus.
% 
% \subsection*{References}
% 
% References should be cited in numerical order as they appear in text; this will be done automatically via bibtex, e.g. \cite{belkin2002using} and \cite{berard1994embedding,coifman2005geometric}. All references should be included in the main manuscript file.  
% 
% \subsection*{Data Archival}
% 
% PNAS must be able to archive the data essential to a published article. Where such archiving is not possible, deposition of data in public databases, such as GenBank, ArrayExpress, Protein Data Bank, Unidata, and others outlined in the Information for Authors, is acceptable.
% 
% \subsection*{Language-Editing Services}
% Prior to submission, authors who believe their manuscripts would benefit from professional editing are encouraged to use a language-editing service (see list at www.pnas.org/site/authors/language-editing.xhtml). PNAS does not take responsibility for or endorse these services, and their use has no bearing on acceptance of a manuscript for publication. 
% 
% \begin{figure}%[tbhp]
%     \centering
%     \includegraphics[width=.8\linewidth]{frog}
%     \caption{Placeholder image of a frog with a long example caption to show justification setting.}
%     \label{fig:frog}
% \end{figure}
% 
% 
% \begin{SCfigure*}[\sidecaptionrelwidth][t]
%     \centering
%     \includegraphics[width=11.4cm,height=11.4cm]{frog}
%     \caption{This caption would be placed at the side of the figure, rather than below it.}\label{fig:side}
% \end{SCfigure*}
% 
% \subsection*{Digital Figures}
% 
% Only TIFF, EPS, and high-resolution PDF for Mac or PC are allowed for figures that will appear in the main text, and images must be final size. Authors may submit U3D or PRC files for 3D images; these must be accompanied by 2D representations in TIFF, EPS, or high-resolution PDF format.  Color images must be in RGB (red, green, blue) mode. Include the font files for any text. 
% 
% Figures and Tables should be labelled and referenced in the standard way using the \verb|\label{}| and \verb|\ref{}| commands.
% 
% Figure \ref{fig:frog} shows an example of how to insert a column-wide figure. To insert a figure wider than one column, please use the \verb|\begin{figure*}...\end{figure*}| environment. Figures wider than one column should be sized to 11.4 cm or 17.8 cm wide. Use \verb|\begin{SCfigure*}...\end{SCfigure*}| for a wide figure with side captions.
% 
% \subsection*{Tables}
% In addition to including your tables within this manuscript file, PNAS requires that each table be uploaded to the submission separately as a “Table” file.  Please ensure that each table .tex file contains a preamble, the \verb|\begin{document}| command, and the \verb|\end{document}| command. This is necessary so that the submission system can convert each file to PDF.
% 
% \subsection*{Single column equations}
% 
% Authors may use 1- or 2-column equations in their article, according to their preference.
% 
% To allow an equation to span both columns, use the \verb|\begin{figure*}...\end{figure*}| environment mentioned above for figures.
% 
% Note that the use of the \verb|widetext| environment for equations is not recommended, and should not be used. 
% 
% \begin{figure*}[bt!]
%     \begin{align*}
%         (x+y)^3&=(x+y)(x+y)^2\\
%             &=(x+y)(x^2+2xy+y^2) \numberthis \label{eqn:example} \\
%             &=x^3+3x^2y+3xy^3+x^3. 
%     \end{align*}
% \end{figure*}
% 
% 
% \begin{table}%[tbhp]
%     \centering
%     \caption{Comparison of the fitted potential energy surfaces and ab initio benchmark electronic energy calculations}
%     \begin{tabular}{lrrr}
%         Species & CBS & CV & G3 \\
%         \midrule
%         1. Acetaldehyde & 0.0 & 0.0 & 0.0 \\
%         2. Vinyl alcohol & 9.1 & 9.6 & 13.5 \\
%         3. Hydroxyethylidene & 50.8 & 51.2 & 54.0\\
%         \bottomrule
%     \end{tabular}
% 
%     \addtabletext{nomenclature for the TSs refers to the numbered species in the table.}
% \end{table}
% 
% \subsection*{Supporting Information (SI)}
% 
% Authors should submit SI as a single separate PDF file, combining all text, figures, tables, movie legends, and SI references.  PNAS will publish SI uncomposed, as the authors have provided it.  Additional details can be found here: \href{http://www.pnas.org/page/authors/journal-policies}{policy on SI}.  For SI formatting instructions click \href{https://www.pnascentral.org/cgi-bin/main.plex?form_type=display_auth_si_instructions}{here}.  The PNAS Overleaf SI template can be found \href{https://www.overleaf.com/latex/templates/pnas-template-for-supplementary-information/wqfsfqwyjtsd}{here}.  Refer to the SI Appendix in the manuscript at an appropriate point in the text. Number supporting figures and tables starting with S1, S2, etc.
% 
% Authors who place detailed materials and methods in an SI Appendix must provide sufficient detail in the main text methods to enable a reader to follow the logic of the procedures and results and also must reference the SI methods. If a paper is fundamentally a study of a new method or technique, then the methods must be described completely in the main text.
% 
% \subsubsection*{SI Datasets} 
% 
% Supply Excel (.xls), RTF, or PDF files. This file type will be published in raw format and will not be edited or composed.
% 
% 
% \subsubsection*{SI Movies}
% 
% Supply Audio Video Interleave (avi), Quicktime (mov), Windows Media (wmv), animated GIF (gif), or MPEG files and submit a brief legend for each movie in a Word or RTF file. All movies should be submitted at the desired reproduction size and length. Movies should be no more than 10 MB in size.
% 
% 
% \subsubsection*{3D Figures}
% 
% Supply a composable U3D or PRC file so that it may be edited and composed. Authors may submit a PDF file but please note it will be published in raw format and will not be edited or composed.


\matmethods{Please describe your materials and methods here. This can be more than one paragraph, and may contain subsections and equations as required. Authors should include a statement in the methods section describing how readers will be able to access the data in the paper. 

\subsection*{Subsection for Method}
Example text for subsection.
}

\showmatmethods{} % Display the Materials and Methods section

\acknow{Please include your acknowledgments here, set in a single paragraph. Please do not include any acknowledgments in the Supporting Information, or anywhere else in the manuscript.}

\showacknow{} % Display the acknowledgments section

% Bibliography
\bibliography{procedural}

\end{document}
